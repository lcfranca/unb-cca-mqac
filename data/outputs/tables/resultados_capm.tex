\begin{table}[H]
\centering
\begin{threeparttable}
\caption{Resultados da Estimação CAPM --- PETR4}
\label{tab:resultados_capm}
\begin{tabular}{lr}
\toprule
\textbf{Parâmetro} & \textbf{Valor} \\
\midrule
\multicolumn{2}{l}{\textit{Parâmetros da Regressão}} \\
Beta ($\beta$) & 1.4002 \\
Erro-padrão do Beta & 0.0273 \\
p-valor & 0.000000 \\
Alfa ($\alpha$) anualizado & 0.26\% \\
$R^2$ & 51.67\% \\
Observações & 2.462 \\
\midrule
\multicolumn{2}{l}{\textit{Taxas de Retorno}} \\
Taxa Livre de Risco ($R_f$) & 9.45\% a.a. \\
Retorno de Mercado ($R_m$) & 14.56\% a.a. \\
Prêmio de Mercado ($R_m - R_f$) & 5.11\% a.a. \\
\midrule
\multicolumn{2}{l}{\textit{Custo de Capital}} \\
$K_e$ (CAPM) = $R_f + \beta(R_m - R_f)$ & 16.60\% a.a. \\
Retorno Realizado (PETR4) & 16.97\% a.a. \\
\textbf{Mispricing} ($r_{realizado} - K_e$) & \textbf{+0.37\%} \\
\midrule
\multicolumn{2}{l}{\textit{Risco}} \\
Volatilidade Idiossincrática & 31.82\% a.a. \\
\bottomrule
\end{tabular}
\begin{tablenotes}
\footnotesize
\item \textit{Fonte}: Elaboração própria com dados da Brapi API, Yahoo Finance e BCB (2025).
\item \textit{Nota}: Regressão OLS dos retornos em excesso de PETR4 sobre os retornos em excesso do Ibovespa. Período de 2016-01-05 a 2025-11-28. Taxa livre de risco baseada na SELIC média do período.
\end{tablenotes}
\end{threeparttable}
\end{table}
