\begin{table}[H]
\centering
\begin{threeparttable}
\caption{Análise de Cenários --- PETR4}
\label{tab:cenarios_valuation}
\begin{tabular}{lrrr}
\toprule
\textbf{Métrica} & \textbf{Base} & \textbf{Otimista} & \textbf{Pessimista} \\
\midrule
\multicolumn{4}{l}{\textit{Premissas}} \\
Brent (US\$/bbl) & 75 & 90 & 50 \\
Produção (MMboe/d) & 2.8 & 3.5 & 2.5 \\
Reservas (\% var.) & +0\% & +30\% & -20\% \\
ERP & 5.5\% & 4.5\% & 7.0\% \\
\midrule
\multicolumn{4}{l}{\textit{CAPM}} \\
Beta & 1.40 & 1.30 & 1.55 \\
Rf & 9.45\% & 8.95\% & 10.45\% \\
Ke & 17.15\% & 14.80\% & 21.30\% \\
\midrule
\multicolumn{4}{l}{\textit{Valuation}} \\
Valor Justo (R\$) & 29.46 & 44.10 & 23.72 \\
Upside/Downside & -7.3\% & +38.7\% & -25.4\% \\
ICC & 15.89\% & 19.37\% & 15.89\% \\
\midrule
\multicolumn{4}{l}{\textit{Q-VAL Score}} \\
Score (0-100) & 53.2 & 62.8 & 43.0 \\
Recomendação & Neutro & Compra & Venda \\
\bottomrule
\end{tabular}
\begin{tablenotes}
\footnotesize
\item \textit{Fonte}: Elaboração própria.
\item \textit{Nota}: Cenário otimista assume sucesso da Margem Equatorial e recuperação de preços de commodities. Cenário pessimista considera bloqueio regulatório e queda de preços do petróleo. ERP = Equity Risk Premium.
\end{tablenotes}
\end{threeparttable}
\end{table}
